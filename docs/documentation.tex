\documentclass[a4paper,12pt]{report}
\usepackage{titlesec}
\usepackage{graphicx}
\usepackage{fancyhdr}
\usepackage{geometry}
\usepackage{indentfirst}
\usepackage{plantuml}

% chktex-file 44

\newgeometry{
    top=2.5cm,
    bottom=3cm,
    inner=2cm,
    outer=2cm,
}

\pagestyle{fancy}
\fancypagestyle{plain}{}

\lhead{\scriptsize Computing and Information Technology\\
    2022\\
    David Daniel, Pava\\
    Walman: A Virtual Wallet Management System}
\rhead{\includegraphics[width=4cm]{images/logo_upt.jpg}}
\cfoot{- \thepage\ -}
\renewcommand{\headrulewidth}{0pt}

\setlength{\headheight}{44pt}

\fancypagestyle{titlepage}{
    \lhead{\footnotesize Computing and Information Technology\\\textbf{2022}}
    \rhead{\includegraphics[width=4cm]{images/logo_upt.jpg}}
    \cfoot{}
}

\titleformat{\chapter}{\center\normalfont\LARGE\bfseries\MakeUppercase}{\thechapter}{0.5em}{}
\titlespacing*{\chapter}{0pt}{3.5ex plus 1ex minus .2ex}{2.3ex plus .2ex}
\title{\LARGE WALMAN\\{ A Virtual Wallet Management System}}

\titleformat{\section}{\normalfont\large\bfseries\MakeUppercase}{\thesection}{0.5em}{}
\titlespacing*{\chapter}{0pt}{3.5ex plus 1ex minus .2ex}{2.3ex plus .2ex}
\title{\LARGE WALMAN\\{ A Virtual Wallet Management System}}

\author{Candidate: David Daniel, Pava
    \\Coordinator: Assoc. Prof.\ Razvan, Bogdan}

\date{June 2022 Session}

\begin{document}

\begin{titlepage}
    \thispagestyle{titlepage}
    \begin{center}
        \vspace*{10cm}
        \LARGE\textbf{Walman\\A Virtual Wallet Management System}\\
        \vfill
    \end{center}
    \begin{flushleft}
        \large\textbf{Candidate: David Daniel, Pava
            \\Coordinator: Assoc. Prof.\ Razvan, Bogdan}
    \end{flushleft}

\end{titlepage}

\chapter*{Abstract}

\tableofcontents

\chapter{Introduction}
% \begin{figure}[h]
%     \centering
%     \includegraphics[scale=0.1]{images/utp_logo.jpg}
%     \caption{Qr Code}\label{fig:qr}
% \end{figure}

\section{Context}

\par In the last 30 years, the number of tasks that are digitalized has increased
exponentially. The most important part of the security systems of these tasks
is user management and authentication. The password is the most widely spread
form of user authentication and thus is often the prime target of attackers
that want to impersonate someone else.

According to~\cite{systematicAnalysis}, a \textit{``systematic literature
    review in the area of passwords and passwords security''}, there are many
problems with password security and management ranging from weak passwords and
password reuse, to users writing down passwords or sending them through
unsecure channels. Most of these problems according
to~\cite{systematicAnalysis} are solved by using password recommendations. A
good solution to most of these problems is a password management tool. A
password manager is a piece of software designed for generating and managing
passwords, in this way the user can have unique, complex and safely stored
password without having to remember them.

Another great method to better secure you accounts is using a two factor
authentication (2FA) method. These method vary from security questions, to one
time passwords (OTP) sent from the server to the user via email or SMS, to OTPs
generated using specialized algorithms such as: HMAC-based One Time Password
(HOTP)\cite{hotp} and Time Based One Time Password (TOTP)\cite{totp}.

The cryptocurrency market is another area that has seen a considerable
development lately. As of May 2022, the market cap of Bitcoin is around 565
billion USD, and the market cap of Ethereum is around 214 billion USD.\@ In the
case of Bitcoin, that is more than double of what it was in 2019 (around 211
billion USD), referenced in~\cite{cryptocurrencyMarketAnalysis}.
Cryptocurrencies also offer secure and long term storage capabilities thanks to
the blockchain technology. Blockchain backups, thanks to the decentralized
nature of the blockchain, are very hard to be tempered with. A traditional
cloud backup could be lost or inaccessible in more than one situations. The
most obvious one is data loss happening as result of a cyber attack or the
company simply going bankrupt. There are also situations in which the company
itself can refuse to serve you anymore. They can freeze your account or just
refuse to serve an entire country all-together, we have the recent example of
companies like Visa and Mastercard refusing to serve russian citizens as result
of political tensions. All these scenarios cannot happen in a decentralized
blockchain system.

Businesses that were traditionally not online like shopping also have inversely
digitalized. Nowadays most of the hypermarkets offer fidelity cards. Usually
these cards are built around a unique barcode or qr code. Often it's hard to
manage all your cards, so a digital storage solution to solve this issue would
help the end user better manage their cards.

Considered all mentioned above, a user has to remember and manage a lot of
information in order to interact with the currently available online
infrastructure. A tool that could help them manage all this data better is a
wallet manager.

\section{Motivation}

My personal motivation for creating a wallet manager is the fact that I want to
use it myself. Also I wanted for a long time to explore the state of the art in
smart contract development, so this was a great occasion to do so.

I chose to create this project in the form of a mobile application since people
tend to have their smartphones with them most of the time, so having a virtual
wallet on your mobile device makes sense.

Another factor that motivates me is the fact that currently in the mobile
application market there are almost no free and open source password management
applications available. The user needs to \textit{trust} the creators of the
application with their data, not knowing how the implementation of the product
is made, they have no guarantee that the data is safe.

\section{Similar Products Available on the Market}

There are a lot of password managers available on the market. In this section
we are going to try to make a comparison between some of the most popular
options available.

\begin{table}[h!]
    \centering
    \begin{tabular}{ | l | l | l | l | l | l | }
        \hline
        \textbf{Property}           & \textbf{LastPass} & \textbf{RememBear} & \textbf{KeePass} & \textbf{PassMan} & \textbf{KeyBase} \\
        \hline
        \textbf{Mobile Version}     & Yes               & Yes                & No               & Yes              & Yes              \\
        \hline
        \textbf{Blockchain Storage} & No                & No                 & No               & No               & Yes              \\
        \hline
        \textbf{Price}              & \$3/month         & \$6/month          & Free             & Free             & Free             \\
        \hline
        \textbf{License}            & Proprietary       & Proprietary        & GPL-2.0          & AGPL-3.0         & BSD-3            \\
        \hline
    \end{tabular}
    \caption{A comparison between some of the most popular password managers.}\label{tab:otherProducts}
\end{table}

First off we have LastPass and RememBear, two very similar password managers,
both having a free and a payed plan. Neither of these two is open source, so
the most pressing issue regarding them is the guarantee that your data is safe.
Without having the ability to see how your data is managed and stored you
cannot be certain that it is secure. Also these applications do not have
blockchain backups.

KeePass is probably the most popular password manager for desktop. It is free
and open source, and the code was analyzed and certified by specialized
organizations such as the Open Source Initiative. The biggest drawback to
KeePass is the aged user interface and the missing mobile application
counterpart. Nowadays a lot of the situations where a user needs access to
their credentials are happening while using smartphones. Also the features are
limited, KeePass doing one thing and doing it well that being password
management. There are no cloud or blockchain backups, so the user needs to
manager backing up and storing their password database themselves.

Similar with KeePass, PassMan is a free and open source password manager. They
have a mobile version of the application, but blockchain backups are missing.
Also, again, PassMan is just a password manager. It does not manage shopping
cards or crypto wallets.

Last but not least there is KeyBase which is not technically a password
manager. KeyBase is a blockchain, decentralized, social media application. You
can store password and secure notes inside the application but from the user
experience point of view, KeyBase was never designed to be a password or wallet
manager. The reason why it is mentioned, is because KeyBase is implemented on
the blockchain, all user data is encrypted and it's free and open source.

\chapter{Technology Stack}

\section{Frontend}

\subsection{Flutter}

Flutter is a mobile application development framework developed by Google in
the Dart programming language. It was released in May 2017 and it currently is
one of the most popular mobile development frameworks.

\subsection{Dart}\label{chapter:dart}

\subsection{Code Generation}

\subsection{Redux}

\section{Firebase}

\subsection{User Management}

\subsection{Firestore}

\section{Blockchain}

\subsection{Ethereum}

\subsection{Smart Contracts}

\subsection{Solidity}

\subsection{Test Networks}

\section{Development Environment}

\chapter{Implementation}

\section{Use Cases}

\section{System Architecture}

\section{Password Management}

\section{QR and Barcode Management}

\section{OTP Authenticator}

\subsection{HOTP}

\subsection{TOTP}

\section{Cryptocurrency Wallet}

\section{Backup}

\subsection{Cloud Backup}

\subsection{Blockchain Backup}

\section{Security}

\chapter{Tests}

\section{Test Pipeline}

\section{Unit Tests}

\section{Widget Tests}

\section{Performance Statistics}

\chapter{Conclusions}

\section{Possible Improvements}

\bibliographystyle{plain}
\bibliography{references}
\addcontentsline{toc}{chapter}{Bibliography}

\end{document}